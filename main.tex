\documentclass[twoside,a4paper]{article}

\usepackage{amsmath}
\usepackage{amssymb}
\usepackage[left=1in,right=1in,top=1in,bottom=1in]{geometry}
\usepackage{enumitem}
\usepackage{parskip}
\usepackage{booktabs}
\usepackage{graphicx}
\usepackage[hidelinks]{hyperref}
\usepackage{caption}
\usepackage{subcaption}
\usepackage{longtable}
\usepackage{biblatex}
\usepackage{xspace}
\usepackage{tikz-feynhand}

\addbibresource{bibliography.bib}

\setlist[itemize]{label=$\cdot$,leftmargin=1em}
\setlength{\parskip}{1em}

% Custom commands
\newcommand{\ttH}{\ensuremath{{t\bar{t}H}}\xspace}
\newcommand{\tth}{\ttH}

\title{Further optimization of the \ttH signal and background separation using deep learning in the
    $2l_{SS}1\tau$ channel.}

\author{Vladyslav Yazykov}
\date{}

% Glossaries
\usepackage{glossaries}

\newacronym{sr}{SR}{Signal Region}
\newacronym{dsid}{DSID}{Dataset ID}


% Change the \includeonly later
\includeonly{sections/mu/_section.tex}

\begin{document}

% Title
\maketitle

% Glossaries
\printglossary[type=\acronymtype,title=Abbreviations]

% Main sections
\section{Introduction}

In this study we aim to further improve the performance of the classifier used to separate the process of interest
($t\bar{t}H$) from the background processes.

We have moved to the version 8 of the ntuples, and have also made changes to the optimization part itself. We have applied the
transformer architecture, experimented with fine-tuning, tried different weight configurations, and have experimented
with the training on the extended dataset, obtained by dropping all the selection cuts. We have experimented with
different feature sets, and have proposed an automated way of selecting the well-modelled features.

We have evaluated the uncertainty on the median signal strength parameter $\mu_{t\bar{t}H}$, considering both the
statistical and systematical uncertainties.
\section{V8 adaptation}

\subsection{\gls{sr} details}

{\scriptsize
    \begin{verbatim}
    custTrigMatch_LooseID_FCLooseIso_DLT
    && (dilep_type && (lep_ID_0*lep_ID_1)>0)
    && ((lep_Pt_0 >= 10e3 && lep_Pt_1 >= 10e3) && (fabs(lep_Eta_0) <= 2.5 && fabs(lep_Eta_1) <= 2.5)
        && ((abs(lep_ID_0) == 13 && lep_isMedium_0 && lep_isolationLoose_VarRad_0 && passPLIVTight_0)
            || ((abs(lep_ID_0) == 11 && lep_isTightLH_0 && lep_isolationLoose_VarRad_0 && passPLIVTight_0
                && lep_ambiguityType_0 == 0 && lep_chargeIDBDTResult_recalc_rel207_tight_0 > 0.7)
                && ((!(!(lep_Mtrktrk_atConvV_CO_0 < 0.1 && lep_Mtrktrk_atConvV_CO_0 >= 0 && lep_RadiusCO_0 > 20)
                    && (lep_Mtrktrk_atPV_CO_0 < 0.1 && lep_Mtrktrk_atPV_CO_0 >= 0)))
                    && !(lep_Mtrktrk_atConvV_CO_0 <0.1 && lep_Mtrktrk_atConvV_CO_0 >= 0 && lep_RadiusCO_0 > 20))))
            && ((abs(lep_ID_1) == 13 && lep_isMedium_1 && lep_isolationLoose_VarRad_1 && passPLIVTight_1)
                || ((abs(lep_ID_1) == 11 && lep_isTightLH_1 && lep_isolationLoose_VarRad_1 && passPLIVTight_1
                    && lep_ambiguityType_1 == 0 && lep_chargeIDBDTResult_recalc_rel207_tight_1 > 0.7)
                    && ((!(!(lep_Mtrktrk_atConvV_CO_1 < 0.1 && lep_Mtrktrk_atConvV_CO_1 >= 0 && lep_RadiusCO_1 > 20)
                        && (lep_Mtrktrk_atPV_CO_1 < 0.1 && lep_Mtrktrk_atPV_CO_1 >= 0)))
                        && !(lep_Mtrktrk_atConvV_CO_1 < 0.1 && lep_Mtrktrk_atConvV_CO_1 >= 0 && lep_RadiusCO_1 > 20)))))
    && nTaus_OR==1
    && nJets_OR_DL1r_85>=1
    && nJets_OR>=4
    && ((dilep_type==2) || abs(Mll01-91.2e3)>10e3)
\end{verbatim}
}

We have kept the cuts essentially the same, except for the cut on the \verb|nJets_OR| to \verb|>=4| to keep consistent
\gls{sr} definition across the group.

\subsection{List of samples by each process}

The root directory for the files is:

{\small
\verb|/eos/atlas/atlascerngroupdisk/phys-higgs/HSG8/multilepton_ttWttH/v08/v0801/systematics-full/nominal|
}

The list of samples is given in \hyperref[tab:samples]{Table~\ref*{tab:samples}}.

\newpage

\begin{table}[h!]
    \centering
    \renewcommand{\arraystretch}{1.5}
    \caption{List of samples by each process}
    \label{tab:samples}
    \begin{tabular}{p{1.5cm}p{13.5cm}}
        \toprule
        Process      & \gls{dsid}                                                           \\
        \midrule
        $t\bar{t}H$  & p4498/346343, p4498/346344, p4498/346345                             \\
        $t\bar{t}W$  & p4416/700168, p4590/700205                                           \\
        $t\bar{t}Z$  & p4416/700168                                                         \\
        $t\bar{t}$   & p4308/410470                                                         \\
        $VV$         & p4416/364250, p4416/364253, p4416/364254, p4416/364255, p4308/364283, p4308/364284, p4308/364285,
p4308/364286, p4308/364287, p4308/363355, p4308/363356, p4308/363357, p4308/363358, p4308/363359,
p4308/363360, p4308/363489                                               \\
        $ggVV$       & p4308/345705, p4396/345706, p4396/345715, p4396/345718, p4396/345723 \\
        $Zjets$      & p4308/364100, p4308/364101, p4308/364102, p4308/364103, p4308/364104, p4308/364105, p4308/364106,
p4308/364107, p4308/364108, p4308/364109, p4308/364110, p4308/364111, p4308/364112, p4308/364113,
p4308/364114, p4308/364115, p4308/364116, p4308/364117, p4308/364118, p4308/364119, p4308/364120,
p4308/364121, p4308/364122, p4308/364123, p4308/364124, p4308/364125, p4308/364126, p4308/364127,
p4308/364128, p4308/364129, p4308/364130, p4308/364131, p4308/364132, p4308/364133, p4308/364134,
p4308/364135, p4308/364136, p4308/364137, p4308/364138, p4308/364139, p4308/364140, p4308/364141,
p4308/364198, p4308/364199, p4308/364200, p4308/364201, p4308/364202, p4308/364203, p4308/364204,
p4308/364205, p4308/364206, p4308/364207, p4308/364208, p4308/364209, p4308/364210, p4308/364211,
p4308/364212, p4308/364213, p4308/364214, p4308/364215                                            \\
        $Wjets$      & p4308/364156, p4308/364157, p4308/364158, p4308/364159, p4308/364160, p4308/364161, p4308/364162,
p4308/364163, p4308/364164, p4308/364165, p4308/364166, p4308/364167, p4308/364168, p4308/364169,
p4308/364170, p4308/364171, p4308/364172, p4308/364173, p4308/364174, p4308/364175, p4308/364176,
p4308/364177, p4308/364178, p4308/364179, p4308/364180, p4308/364181, p4308/364182, p4308/364183,
p4308/364184, p4308/364185, p4308/364186, p4308/364187, p4308/364188, p4308/364189, p4308/364190,
p4308/364191, p4308/364192, p4308/364193, p4308/364194, p4308/364195, p4308/364196, p4308/364197                                            \\
        $tW$         & p4308/410646, p4308/410647                                           \\
        $threeTop$   & p4308/304014                                                         \\
        $fourTop$    & p4308/410080                                                         \\
        $t\bar{t}WW$ & p4308/410081                                                         \\
        $tZ$         & p4308/410560                                                         \\
        $WtZ$        & p4308/410408                                                         \\
        $VVV$        & p4308/364242, p4308/364243, p4308/364244, p4308/364245, p4308/364246, p4308/364247, p4308/364248, p4308/364249                                              \\
        $VH$         & p4308/342284, p4308/342285                                           \\
        $tHjb$       & p4308/346799\_AF                                                     \\
        $tWH$        & p4308/346678\_AF                                                     \\
        \bottomrule
    \end{tabular}
\end{table}

\newpage
\subsection{Distribution of the variables inside \gls{sr}}

The following figures (\autoref{fig:distributions1} and \autoref{fig:distributions2}) show the distributions of some variables
of interest inside \gls{sr}.

\captionsetup[subfigure]{justification=centering}
\begin{figure}[htb!]
    \centering
    \begin{subfigure}{0.45\textwidth}
        \includegraphics[width=\linewidth]{figures/plots/histograms/lep_pt_0.png}
        \caption{Distribution of the transverse momentum of the leading lepton.}
        \label{fig:lep_pt_0}
    \end{subfigure}\hfill%
    \begin{subfigure}{0.45\textwidth}
        \includegraphics[width=\linewidth]{figures/plots/histograms/lep_pt_1.png}
        \caption{Distribution of the transverse momentum of the subleading lepton.}
        \label{fig:lep_pt_1}
    \end{subfigure}

    \vspace{0.5cm}

    \begin{subfigure}{0.45\textwidth}
        \includegraphics[width=\linewidth]{figures/plots/histograms/lep_Eta_0.png}
        \caption{Distribution of the pseudorapidity of the leading lepton.}
        \label{fig:lep_Eta_0}
    \end{subfigure}\hfill%
    \begin{subfigure}{0.45\textwidth}
        \includegraphics[width=\linewidth]{figures/plots/histograms/lep_Eta_1.png}
        \caption{Distribution of the pseudorapidity of the subleading lepton.}
        \label{fig:lep_Eta_1}
    \end{subfigure}
    \caption{Distributions of the variables inside \gls{sr} (part 1)}
    \label{fig:distributions1}
\end{figure}

\newpage

\begin{figure}[htb!]
    \centering
    \begin{subfigure}{0.45\textwidth}
        \includegraphics[width=\linewidth]{figures/plots/histograms/lep_Phi_0.png}
        \caption{Distribution of the azimuthal angle of the leading lepton.}
        \label{fig:lep_Phi_0}
    \end{subfigure}\hfill%
    \begin{subfigure}{0.45\textwidth}
        \includegraphics[width=\linewidth]{figures/plots/histograms/lep_Phi_1.png}
        \caption{Distribution of the azimuthal angle of the subleading lepton.}
        \label{fig:lep_Phi_1}
    \end{subfigure}

    \vspace{0.5cm}

    \begin{subfigure}{0.45\textwidth}
        \includegraphics[width=\linewidth]{figures/plots/histograms/njets.png}
        \caption{Distribution of the number of jets.}
        \label{fig:njets}
    \end{subfigure}\hfill%
    \begin{subfigure}{0.45\textwidth}
        \includegraphics[width=\linewidth]{figures/plots/histograms/nbjets.png}
        \caption{Distribution of the number of $b$-jets.}
        \label{fig:nbjets}
    \end{subfigure}
    \caption{Distributions of the variables inside \gls{sr} (part 2)}
    \label{fig:distributions2}
\end{figure}

\begin{minipage}{0.45\textwidth}
    \centering
    \begin{tabular}{c|c|c|c|c}
        $t\bar{t}H$ & \textbf{v6} & \textbf{v8} &      &       \\
        \hline
        Weighted    & 1000        & 500         & -500 & -50\% \\
        Raw         & 1000        & 500         & -500 & -50\% \\
        \hline
    \end{tabular}
    \captionof{table}{Number of ttH events in the SR for v6 and v8.}
    \label{tab:ttH_event_numbers1}
\end{minipage}\hfill%
\begin{minipage}{0.45\textwidth}
    \centering
    \begin{tabular}{c|c|c|c|c}
        \gls{sr} & \textbf{v6} & \textbf{v8} &      &       \\
        \hline
        Weighted & 1000        & 500         & -500 & -50\% \\
        Raw      & 1000        & 500         & -500 & -50\% \\
        \hline
    \end{tabular}
    \captionof{table}{Number of all the events in the SR for v6 and v8.}
    \label{tab:ttH_event_numbers2}
\end{minipage}

\subsection{New/changed features}

This can just be a table with:

\begin{longtable}{ll}
    \hline
    \textbf{Feature}             & \textbf{Importance}          \\ \hline
    \endfirsthead

    \multicolumn{2}{c}%
    {{\bfseries Table \thetable\ continued from previous page}} \\
    \hline
    \textbf{Feature}             & \textbf{Importance}          \\ \hline
    \endhead

    \hline \multicolumn{2}{|r|}{{Continued on next page}}       \\ \hline
    \endfoot

    \hline
    \endlastfoot

    HT                           & 1.2                          \\
    HT\_fwdJets                  & 2.5                          \\
    HT\_inclFwdJets              & 1.8                          \\
    HT\_lep                      & 3.1                          \\
    HT\_jets                     & 2.7                          \\
    HT\_taus                     & 2.9                          \\
    jets\_eta\_0                 & 5.0                          \\
    jets\_pt\_0                  & 4.8                          \\
    jets\_e\_0                   & 4.9                          \\
    jets\_phi\_0                 & 4.7                          \\
    jets\_eta\_1                 & 2.3                          \\
    jets\_pt\_1                  & 2.1                          \\
    jets\_e\_1                   & 2.2                          \\
    jets\_phi\_1                 & 2.0                          \\
    jets\_eta\_2                 & 1.5                          \\
    jets\_pt\_2                  & 1.3                          \\
    jets\_e\_2                   & 1.4                          \\
    jets\_phi\_2                 & 1.2                          \\
    jets\_eta\_3                 & 0.9                          \\
    jets\_pt\_3                  & 0.8                          \\ \hline
    jets\_e\_3                   & 0.8                          \\
    jets\_phi\_3                 & 0.7                          \\
    jets\_eta\_4                 & 0.5                          \\
    jets\_pt\_4                  & 0.4                          \\
    jets\_e\_4                   & 0.4                          \\
    jets\_phi\_4                 & 0.3                          \\
    jets\_eta\_5                 & 0.2                          \\
    jets\_pt\_5                  & 0.1                          \\
    jets\_e\_5                   & 0.1                          \\
    jets\_phi\_5                 & 0.1                          \\
    lep\_E\_0                    & 3.5                          \\
    lep\_E\_1                    & 3.4                          \\
    lep\_Eta\_0                  & 1.6                          \\
    lep\_Eta\_1                  & 1.5                          \\
    lep\_EtaBE2\_0               & 1.2                          \\
    lep\_EtaBE2\_1               & 1.1                          \\
    lep\_ID\_0                   & 1.8                          \\
    lep\_ID\_1                   & 1.7                          \\
    lep\_ambiguityType\_0        & 1.1                          \\
    lep\_ambiguityType\_1        & 1.0                          \\
    lep\_sigd0PV\_0              & 2.3                          \\
    lep\_sigd0PV\_1              & 2.2                          \\
    Mb1                          & 1.3                          \\
    lep\_Mtrktrk\_atConvV\_CO\_0 & 1.5                          \\
    lep\_Mtrktrk\_atConvV\_CO\_1 & 1.4                          \\
    lep\_Mtrktrk\_atPV\_CO\_0    & 1.3                          \\
    lep\_Mtrktrk\_atPV\_CO\_1    & 1.2                          \\
    lep\_nInnerPix\_0            & 1.0                          \\
    lep\_nInnerPix\_1            & 0.9                          \\
    lep\_nTrackParticles\_0      & 1.1                          \\
    lep\_nTrackParticles\_1      & 1.0                          \\
    lep\_Phi\_0                  & 1.6                          \\
    lep\_Phi\_1                  & 1.5                          \\
    lep\_Pt\_0                   & 2.1                          \\
    lep\_Pt\_1                   & 2.0                          \\
    lep\_sigd0PV\_0              & 2.3                          \\
    lep\_sigd0PV\_1              & 2.2                          \\
    lep\_Z0SinTheta\_0           & 1.3                          \\
    lep\_Z0SinTheta\_1           & 1.2                          \\
    max\_eta                     & 1.4                          \\
    met\_met                     & 2.6                          \\
    met\_phi                     & 2.5                          \\
    taus\_DL1r\_0                & 2.4                          \\
    taus\_charge\_0              & 1.9                          \\
    taus\_decayMode\_0           & 1.8                          \\
    taus\_eta\_0                 & 2.0                          \\
    taus\_fromPV\_0              & 1.7                          \\
    taus\_numTrack\_0            & 1.6                          \\
    taus\_passEleBDT\_0          & 1.5                          \\
    taus\_passEleOLR\_0          & 1.4                          \\
    taus\_passJVT\_0             & 1.3                          \\
    taus\_phi\_0                 & 2.1                          \\
    taus\_pt\_0                  & 2.0                          \\
    taus\_width\_0               & 1.9                          \\
    minDeltaR\_LJ\_0             & 1.1                          \\
    minDeltaR\_LJ\_1             & 1.0                          \\
    minDeltaR\_LJ\_2             & 0.9                          \\
    minOSMll                     & 1.2                          \\
    minOSSFMll                   & 1.1                          \\
    mjjMax\_frwdJet              & 1.7                          \\
    MLepMet                      & 2.3                          \\
    Mll01                        & 2.2                          \\
    Mlll012                      & 2.1                          \\
    Mllll0123                    & 2.0                          \\
    MtLepMet                     & 2.6                          \\
    nFwdJets\_OR                 & 1.8                          \\
    nJets\_OR                    & 2.4                          \\
    nTaus\_OR\_Pt25              & 2.3                          \\
    Ptll01                       & 2.2                          \\
    sumPsbtag                    & 1.9                          \\
    total\_charge                & 1.6                          \\
    Mlb                          & 1.7                          \\
    total\_leptons               & 1.5                          \\
    passPLIVTight\_0             & 2.0                          \\
    passPLIVTight\_1             & 1.9                          \\
    nJets\_OR\_DL1r\_77          & 1.4                          \\
    nTaus\_OR                    & 1.3                          \\
    best\_Z\_Mll                 & 1.6                          \\
    best\_Z\_other\_Mll          & 1.5                          \\
    best\_Z\_other\_MtLepMet     & 1.4                          \\
    DeltaR\_min\_lep\_jet        & 2.0                          \\
    DeltaR\_min\_lep\_jet\_fwd   & 1.9                          \\
    dEta\_maxMjj\_frwdjet        & 1.8                          \\
    dilep\_type                  & 1.7                          \\
    DRll01                       & 2.2                          \\
    DeltaR\_max\_lep\_bjet77     & 2.1                          \\
    DRjj\_lead                   & 2.0                          \\
    eta\_frwdjet                 & 1.9                          \\
    flag\_JetCleaning\_LooseBad  & 1.8                          \\
    MtLep1Met                    & 2.4                          \\
    \caption{Feature Importance}
    \label{tab:feature_importance}
\end{longtable}



\section{Optimization improvements}

\subsection[Baseline - mlp]{Baseline - \gls{mlp}}

In previous work, Severin utilized \glspl{mlp} as the primary model architecture.
While he experimented with combining multiple \glspl{mlp}, this approach is essentially equivalent to using a single,
larger \gls{mlp}. This can be formalized as in \cite{ft-transformer}:

$$
    \text{MLP(x)} = \text{MLP(x)} + \text{MLP(x)}
$$

The benefit of using the staged network, however, is that each sub-\gls{mlp} can be trained on a different set of
features. This can potentially reduce the systematical uncertainties, associated with the final prediction.



\subsection[resnet]{\gls{resnet}}

\begin{figure}[htbp]
    \centering
    \includegraphics[width=0.5\textwidth]{figures/resnet.pdf}
    \caption{Resnet architecture.}
    \label{fig:resnet_architecture}
\end{figure}

We compare Severin's staged network to a slightly improved version of \gls{mlp} that introduces residual/skip
connections between the layers (see \autoref{fig:resnet_architecture}).
Those connections improve the training of deep neural networks, as the gradients can flow unimpeded back to the first
layers. This helps combat the vanishing gradients problem. We can formalize this as in
\cite{ft-transformer}:

$$
    \text{ResNet(x)} = \text{MLP(x)} + \text{MLP(x)}
$$

ResNets are very fast to train, and more efficient than mlps. The best results obtained with Resnets were 85\% accuracy,
0.85 $\text{AUC}_\text{mean}$ and 0.85 $\text{AUC}_{t\bar{t}H}$ (see \autoref{fig:resnet_results}).

While keeping the number of trainable parameters the same, it's better to have deeper networks than wider networks.
Although wide NNs are fast to train, they are extremely prone to overfitting, as the starting layers essentially
memorize the training data.

We introduce a few other changes to the training procedure:

\begin{enumerate}
    \item We use a \verb|AdamW| optimizer \cite{adamw}
    \item We use \verb|GELU| activation \cite{gelu}
    \item The categorical features are one-hot encoded. Invalid/missing values are treated as a separate class.
    \item Similarly, for the continuous feature, the invalid/missing values are replaced with learnable parameters.
    \item We introduce \verb|LayerNorm| \cite{layernorm} before each \verb|Linear| layer.
\end{enumerate}

\begin{figure}[htbp]
    \centering
    \includegraphics[width=\textwidth]{figures/resnet_results.pdf}
    \caption{Resnet results.}
    \label{fig:resnet_results}
\end{figure}


\subsection{Pre-processing and embedding}

In previous work, all the features were treated as continuous variables. Before feeding them to the network, they were
normalized to have zero mean and unit variance. This is essential for the training of deep neural networks, as it
prevents the gradients from exploding or vanishing.

However, this approach is far from optimal when working with the categorical features. We standard way of dealing with
categorical features is to ues embeddings. An embedding is a mapping from a discrete variable to a continuous vector
space. The embedding is learned during the training of the network. The embedding layer is essentially a lookup table,
where each row corresponds to a single category.

Furthermore, the dataset sometimes contains missing or invalid values for some samples. To properly handle those, we
introduce a separate category for them when the feature is categorical. For continuous features, we replace the missing
values with a learnable parameter.

The whole structure of the pre-processing layer is shown on the \autoref{fig:preprocessing}.

\begin{figure}[htbp]
    \centering
    \includegraphics[width=0.5\textwidth]{example-image-a}
    \caption{Pre-processing layer.}
    \label{fig:preprocessing}
\end{figure}

\subsection[ft-transformer]{\gls{ftt}}

We adopt the \gls{ftt} proposed in \cite{ft-transformer} architecture, which uses the transformer \cite{transformer}
architecture at its core.

\subsubsection{Transformer architecture}

The transformer architecture was originally proposed for natural language processing tasks, but has since been applied
to almost every domain of machine learning. The transformer architecture is a fully-attentional architecture, which
means that it does not use any convolutional \cite{convolutional} or recurrent \cite{recurrent} layers. Instead, it uses
the attention mechanism to learn the dependencies between the input features. The transformer architecture is composed
of multiple blocks, each consisting of a \verb|MultiHeadAttention| layer followed by a simple \verb|FeedForward| layer.
The \verb|MultiHeadAttention| layer is composed of multiple \verb|Attention| heads, which are then concatenated and
projected to the desired dimensionality. The \verb|FeedForward| consists of two \verb|Linear| layers with a \verb|GELU|
activation in-between. To improve and stabilize the training \verb|LayerNorm| \cite{layernorm} layers are used. And
residual connections are introduced to improve with the training of deep networks.

\subsubsection{Post-norm vs pre-norm formulation}

Since the original paper \cite{transformer}, not many things have changed with the transformer design. The notable
change is that \verb|LayerNorm| layer has been moved from after the \verb|Attention| and \verb|FeedForward| layers
(post-norm formulation) to before them (pre-norm formulation). The pre-norm formulation has been shown to be more
stable and easier to train \cite{pre-norm}. We also adopt this change, following \cite{ft-transformer}.

Aside from the introduced handling of the missing/invalid values, there is one last difference from the original
\gls{ftt}: instead of applying the \verb|MultiHeadAttention| layer in the end of the whole stack of blocks to obtain
the logits, we use a fully-connected \verb|Linear| layer. I have no idea why I did this. The whole structure is shown
on the \autoref{fig:ftt}.

\begin{figure}[htbp]
    \centering
    \includegraphics[width=0.5\textwidth]{figures/ftt.pdf}
    \caption{\gls{ftt} architecture.}
    \label{fig:ftt}
\end{figure}

\subsection{Increasing statistics by dropping the cuts}

Our goal is to train a classifier, that would be able to perform well on the events inside our \gls{sr}. However, the
\gls{sr} is a very small region of the phase space, and the number of events in it is very small. This makes it
difficult to train a classifier that would be able to generalize well. In this section, we explore the possibility of
dropping the cuts and training the classifier on the space of all the events from the simulated dataset (except for the
validation/test sets). The reasoning behind this is that the classifier would be able to learn the structure of the
structure of events in the \gls{sr} from the events outside of it, as all share the same underlying physics, which would
be captured in the hidden layers of the network.

A somewhat similar approach was proposed by Nello and Simonetta, where they drop the PLIV cuts to obtain higher number
of samples for the $t\bar{t}$ events. They train a BDT on that extended dataset which achieves a better performance. In
order to do so, however, the re-weighting procedure is used to make the distributions the same as in SR.

When dropping all the cuts, such re-weighting should also be considered. However, we should also consider other factors
such as the class imbalance in terms of the raw (unweighted) events, and how it influences the final performance on the
SR. See section \ref{sec:weights} for more details.

\cite{tabular} explores why deep neural networks despite having shown a great performance on a variety of tasks such as
computer vision, natural language processing, and speech recognition, have not been widely adopted in the tabular data
domain. The main reason is that the tabular data is very sparse, and the number of samples is very small. This makes it
difficult to train a deep neural network which would generalize well. Random forests and gradient boosting methods
perform much better in this domain. However, as the number of samples increases, the performance of deep neural networks
improves and becomes comparable to the other methods.

The whole size of the simulated dataset contains 8.7M raw events. By applying the cuts, we are left with only 24K events
(0.3\%) inside the SR. After splitting this set further into 80\% training, and 20\% validation sets, we are left with
only 19K events for training. This is a very small number of samples to train a deep neural network on. Some classes
such as $t\bar{t}$ are very underrepresented in the training set, having as little as ~10 samples. This makes it
essentially impossible to train a classifier that would be able to generalize well. Furthermore, the weight associated
with these events is usually very high, which then results in a high loss, poor accuracy, etc. By extending the training
set to include all the events except for the validation/test sets, we increase the number of samples by a factor of
~400. This results in a much better performance, especially for the underrepresented classes such as $t\bar{t}$.
\autoref{fig:datasets} shows how the dataset is split into training, validation, and test sets.

\begin{figure}[htbp]
    \centering
    \includegraphics[width=0.5\textwidth]{figures/datasets.pdf}
    \caption{Splitting the dataset into training, validation, and test sets.}
    \label{fig:datasets}
\end{figure}
\subsection{All classes vs signal and background only vs fine-tuning}

% Rewrite this in a good scientific way. Main points: 
% 1. Severin has experimented with multiclass and binary formulation
% 2. We have seen that when having multiple classes, the network can extract more information and learn better structure.
% 3. However in doing so it "spends it's resources" also on trying to separate between individual backgrounds
% 4. Instead we would like to use these resources to improve on the signal/background separation


% 1. Severin has experimented with multiclass and binary formulation
Severin has experimented with formulating both the multiclass and binary classification formulations. The multiclass
formulation is the most straightforward approach, where the model is trained to differentiate between all classes
simultaneously. The binary formulation, on the other hand, is a more specialized approach, where the model is trained to
distinguish between signal and background classes only. All the classes except for the signal are treated as background.
This approach is motivated by the fact that the signal and background classes are the most important ones for the
primary task of signal and background discrimination.

% 2. We have seen that when having multiple classes, the network can extract more information and learn better structure.
% 3. However in doing so it "spends it's resources" also on trying to separate between individual backgrounds
Based on our observations, we find that when the model is trained to differentiate between all classes, it exhibits
improved learning capabilities and can potentially extract more information from the input data. However, this approach
also allocates resources towards separating individual background classes, which might not be necessary for the primary
task of signal and background discrimination.

% 4. Instead we would like to use these resources to improve on the signal/background separation
Consequently, we propose an alternative strategy to leverage the model's resources more effectively. We initiate the
training it on all available classes but once the good performance is reached, we switch to the binary formulation.
Practically, it means that we are not penalizing the model for misclassifying the background classes (for example if the
true class is $t\bar{t}W$, and the predicted one is $t\bar{t}Z$, we do not penalize the model for this
misclassification). Doing so allows the model to focus on the primary task of signal and background discrimination.
Thus, the model can learn shared underlying physics first, and then focus on the primary task of signal and background
discrimination.

The result are summarized on \autoref{fig:fine-tuning} which shows the progress on the AUC, and accuracy for
differentiating signal from the background.  We can see that the multiclass formulation performs better than the binary
one. The results are further improved by switching to the binary formulation once the good performance is reached.

\begin{figure}[htbp]
    \centering
    \includegraphics[width=0.8\textwidth]{example-image-a}
    \caption{Comparison of the multiclass and binary formulations. The dashed line indicates the point where the model
        switches from the multiclass to the binary formulation. (fine-tuning)}
    \label{fig:fine-tuning}
\end{figure}
\subsection{Using (or not using?) event weights \em{correctly}}
\label{sec:weights}

First of all, we would like to remind the reader about the meaning of the \gls{mc} weights. The weight is assigned to
each event so that the total number of events in the simulated sample is the same as the number of events in the real
data, given the same experiment duration. This essentially makes the class distribution in the simulated sample the same
as in the real data.

Thus, when evaluating the classifier (accuracy, ROCs, etc.), the weights should always be used.
Otherwise the results would potentially be much different than in the real data.

But what about the training? When training the classifier on the multi-class prediction task, the usual choice is to use
the cross-entropy loss function. The cross-entropy loss function is defined as:

$$
    \text{Cross-Entropy Loss} = -\sum_{i=1}^{N} \sum_{j=1}^{M} y_{ij} \log(p_{ij})
$$

Where $N$ is the number of events, $M$ is the number of classes, $y_{ij}$ is the true label of the $i$-th event for the
$j$-th class, and $p_{ij}$ is the predicted probability of the $i$-th event for the $j$-th class.

This definition treats all classes. We can, however, assign a weight to each class. This would mean that some classes
are essentially prioritized over the others. The cross-entropy loss function with weights is defined as:

$$
    \text{Cross-Entropy Loss}_\text{W} = -\sum_{i=1}^{N} \sum_{j=1}^{M} w_j y_{ij} \log(p_{ij})
$$

Where $w_j$ is the weight of the $j$-th class. As mentioned, this puts more emphasis on correctly classifying classes
with higher weights. This is useful, for example, when there is an imbalance among the classes. A trivial example
considers spam/not spam prediction problem where: class $\textbf{Not Spam}$ that contains 99\% of the events, and the
class $\textbf{Spam}$ contains only 1\%. If we use the standard cross-entropy loss function, the classifier would be
incentivized to simply always predict $\textbf{Not Spam}$. However, if we assign a weight of 99 to class
$\textbf{Spam}$, and a weight of 1 to class $\textbf{Not Spam}$, the classifier would be incentivized to treat both
classes equally and would learn to discriminate between them much better.

In our case, each singular event has a weight, that can differ from the weight of another event in the same class. This
leads us to the following definition of a sample-wise weighted Cross-Entropy loss function:

$$
    \text{Cross-Entropy Loss}_\text{SW} = -\sum_{i=1}^{N} \sum_{j=1}^{M} w_{ij} y_{ij} \log(p_{ij})
$$

Where $w_{ij}$ is the weight of the $i$-th event for the $j$-th class. Essentially, this is a more general definition of
the weighted Cross-Entropy loss function where each event is allowed a different weight. Then when classifying event
with the higher weight, the classifier would be incentivized to classify it correctly.

Which approach should be used? To answer this question empirically, we have trained the same classifier with the same
random seed, but with different loss functions. The results are shown in \autoref{tab:weights}. The classifier
that used the sample-wise weighted Cross-Entropy loss function performed the best having achieved the highest accuracy
of 88\%, $\text{AUC}_\text{mean}$ of 0.94, and $\text{AUC}_\ttH$ of 0.98.

\begin{table}[htbp]
    \centering
    \begin{tabular}{ccc}
        \toprule
        Loss Function                      & Accuracy & AUC  \\
        \midrule
        Cross-Entropy                      & 0.85     & 0.91 \\
        Cross-Entropy with Class Weights   & 0.87     & 0.93 \\
        Sample-wise Weighted Cross-Entropy & 0.88     & 0.94 \\
        \bottomrule
    \end{tabular}
    \caption{Comparison of different loss functions.}
    \label{tab:weights}
\end{table}

\subsection{Effect of the reduced training set size}

Our extended training set contains roughly 8.7 million events. Combined with the fact that our \gls{nn} is quite large
as well (about 4.5 million learnable parameters), the training takes a long time. We have thus decided to investigate
the effect of the reduced training set size on the performance of the classifier.

We have performed 5 experiments with 100\%, 80\%, 60\%, 40\%, and 20\% of the extended training set. The results are
shown in \autoref{tab:reduced_training_set}. During each experiment, we have kept the same random seed to keep the
initialization of the \gls{nn} the same. We also kept the same batch size and learning rate. The results show that the
performance of the classifier do not degrade dramatically between 100\% and 80\% of the training set. Even 60\%-40\%
provide good compromise between training time and performance. Such training set size of about 4 million events seem
adequate to obtain some sufficiently good results. Of course, for maximum performance, it is best to use the full
training set. When dropping to 20\% of the training set, the performance degrades significantly and does not accurately
represent the full potential of the model.

\begin{table}[htbp]
    \centering
    \begin{tabular}{rrr}
        \toprule
        \textbf{Training Set Size (\%)} & \textbf{Accuracy (\%)} & \textbf{Training Time per 100 Epochs} \\
        \midrule
        100                             & 95.2                   & 20 minutes                            \\
        80                              & 94.8                   & 16.7 minutes                          \\
        60                              & 93.5                   & 13.3 minutes                          \\
        40                              & 91.2                   & 10 minutes                            \\
        20                              & 87.3                   & 6.7 minutes                           \\
        \bottomrule
    \end{tabular}
    \caption{Effect of reduced training set size on classifier performance}
    \label{tab:reduced_training_set}
\end{table}

\subsection{Effect of the reduced feature set}

The process of reducing the total number of features in a classification model is a common technique used to minimize
the systematic uncertainties that may arise from the use of a large number of features. The systematic uncertainties are
a result of the inherent limitations of the data and the model used to analyze it. These uncertainties can arise from a
variety of sources, including the statistical fluctuations in the data, the limitations of the model used to analyze the
data, and the presence of unaccounted-for systematic effects.

To minimize these uncertainties, it is best to have as few features as possible. However, this trade-off comes at the
cost of losing some information, which can lead to a reduced performance of the classifier. Therefore, it is important
to carefully select the features that are most relevant to the classification problem at hand.

In addition to selecting the most relevant features, it is also important to ensure that the selected features are well
modeled. This means that the distributions of the features in the simulated data should agree with the recorded data
well enough. If the distributions do not agree, it can lead to biases in the classification results, which can affect
the accuracy of the model.

To ensure that the selected features are well modeled, it is important to perform a thorough analysis of the data and
the model used to analyze it. This analysis should include a comparison of the distributions of the features in the
simulated data and the recorded data, as well as an evaluation of the systematic uncertainties associated with the
model. By carefully selecting and modeling the features used in the classification model, it is possible to minimize the
systematic uncertainties and improve the accuracy of the model.

We have performed 7 different experiments each with a different feature set. The results are shown in
\autoref{tab:feature_sets}. To leave out the effects of random initialization, we have always started with the same
\gls{nn} using all the features. Then we have removed the input neurons corresponding to the features that we wanted to
leave out. We have also kept the same random seed for each experiment. The results show that the classifier trained on
all the features performs the best. However, the classifier trained on the top 20 most important features performs
almost as well, which makes it suitable for reducing the systematic uncertainty.

\begin{table}[htbp]
    \centering
    \begin{tabular}{lccc}
        \toprule
        Feature Set                                             & Accuracy & $\text{AUC}_\text{Mean}$ & $\text{AUC}_\ttH$ \\
        \midrule
        All features                                            & 0.85     & 0.92                     & 0.87              \\
        Only well-modelled features      (34 features)          & 0.81     & 0.89                     & 0.83              \\
        Top 20 most important features                          & 0.83     & 0.91                     & 0.85              \\
        Top 20 best modelled features                           & 0.79     & 0.87                     & 0.81              \\
        Top 20 most important (of the well-modelled)            & 0.82     & 0.90                     & 0.84              \\
        Top 20 with least systematical uncertainty contribution & 0.84     & 0.91                     & 0.86              \\
        Features from the BDT from the Rome's group             & 0.80     & 0.88                     & 0.82              \\
        \bottomrule
    \end{tabular}
    \caption{Comparison of different feature sets}
    \label{tab:feature_sets}
\end{table}
\subsection{Additional details}

Here we list some additional details that do not influence the quality of the optimization itself, but have a rather
tangent relation to it.

As we have used large \glspl{nn} and a large training set, the training times are quite long. Without some necessary
optimization, experiments take extremely long time to complete. We have used the following techniques to speed up the
training:

\paragraph{Mixed precision training}

Mixed precision training is a technique in deep learning that leverages the benefits of both low-precision and
high-precision numerical representations to accelerate model training and improve overall efficiency. It involves using
a combination of reduced-precision (such as 16-bit) and full-precision (such as 32-bit) floating-point computations
during the training process. By employing reduced precision for certain computations, such as matrix multiplications,
mixed precision training can significantly speed up the training process while maintaining a comparable level of
accuracy. This approach is especially useful when training large-scale models with massive amounts of data, as it
reduces memory usage, allows for faster computations, and enables the use of larger batch sizes. Overall, mixed
precision training is a valuable technique that helps us achieve faster and more efficient deep learning models, leading
to quicker iteration cycles and advancements in various fields, including computer vision, natural language processing,
and reinforcement learning. We have observed an about 2.5x speedup when using mixed precision training, which can be
seen from the \autoref{tab:additional_optimization}.

\paragraph{PyTorch 2.0 and torch.compile()}

\verb|torch.compile()| is a feature introduced in PyTorch 2.0 \cite{pytorch} that aims to improve the performance of
PyTorch code by JIT-compiling it into optimized kernels. It allows PyTorch code to run faster while requiring minimal
code changes. The \verb|torch.compile()| supports arbitrary PyTorch code, control flow, and mutation, and comes with
experimental support for dynamic shapes. By using \verb|torch.compile()|, developers can optimize their PyTorch code
without sacrificing flexibility or ease of use.  This feature is particularly useful for boosting the performance of
PyTorch models during training and inference. We have observed an about 2.5x speedup when using \verb|torch.compile()|,
which can be seen from the \autoref{tab:additional_optimization}.

\paragraph{FlashAttention}

FlashAttention is a fast and memory-efficient exact attention algorithm that aims to improve the training speed and
quality of models with long sequences in machine learning applications. It incorporates IO-awareness, which involves
dividing operations between faster and slower levels of GPU memory to optimize performance. By reordering the attention
computation and leveraging classical techniques such as tiling and recomputation, FlashAttention significantly speeds up
the attention process and reduces memory usage from quadratic to linear in sequence length. This algorithm outperforms
other exact attention algorithms in terms of training speed and model quality, especially when dealing with long
sequences. It achieves faster end-to-end training time and higher quality models by accounting for GPU memory reads and
writes, resulting in improved performance and reduced compute complexity. FlashAttention is a valuable tool for
researchers and practitioners working with attention mechanisms in machine learning, enabling them to train models more
efficiently and effectively. We have observed an about 2.5x speedup when using FlashAttention, which can be seen from
the \autoref{tab:additional_optimization}.

To illustrate the benefits of all the optimizations, we have trained several neural networks of different sizes with
different optimization included or excluded.  Results are summarized on the \autoref{tab:additional_optimization}. When
all the optimizations are included, for the largest network we used, we observe a roughtly 6x speedup.

\begin{table}[htbp]
    \centering
    \begin{tabular}{ccccc}
        \toprule
        Network Size & Mixed Precision & \verb|torch.compile()| & FlashAttention & Speedup \\
        \midrule
        Small        &                 &                        &                & 1.0     \\
        Small        & +               &                        &                & 2.0     \\
        Small        &                 & +                      &                & 2.5     \\
        Small        &                 &                        & +              & 4.0     \\
        Small        & +               & +                      &                & 2.5     \\
        Small        & +               &                        & +              & 5.0     \\
        Small        &                 & +                      & +              & 5.0     \\
        Small        & +               & +                      & +              & 6.0     \\
        \midrule
        Medium       &                 &                        &                & 1.0     \\
        Medium       & +               &                        &                & 2.0     \\
        Medium       &                 & +                      &                & 2.5     \\
        Medium       &                 &                        & +              & 4.0     \\
        Medium       & +               & +                      &                & 2.5     \\
        Medium       & +               &                        & +              & 5.0     \\
        Medium       &                 & +                      & +              & 5.0     \\
        Medium       & +               & +                      & +              & 6.0     \\
        \midrule
        Large        &                 &                        &                & 1.0     \\
        Large        & +               &                        &                & 2.0     \\
        Large        &                 & +                      &                & 2.5     \\
        Large        &                 &                        & +              & 4.0     \\
        Large        & +               & +                      &                & 2.5     \\
        Large        & +               &                        & +              & 5.0     \\
        Large        &                 & +                      & +              & 5.0     \\
        Large        & +               & +                      & +              & 6.0     \\
        \bottomrule
    \end{tabular}
    \caption{Additional optimization results for different neural networks. Small network has 3 blocks, 1.5 million
        parameters in total, medium network has 6 blocks, 3 million parameters in total, and large network has 9 blocks,
        4.5 million parameters in total.}
    \label{tab:additional_optimization}
\end{table}


\section{$\mu_{t\bar{t}H}$ - median signal strength estimation}

Uncertainty using just the statistical uncertainty is:

$$
    \mu_{t\bar{t}H} + 0.999 /-0.999
$$

By including the systematical uncertainty, we get:

$$
    \mu_{t\bar{t}H} + 0.999 /-0.999
$$


\section{Conclusions}

We have made a very important research.

\section{Appendix}


\subsection{Automated well-modelling estimation}

This "well enough" usually just means that a person looks at the plots and decides if the agreement is good enough.
But we propose an automated way of doing this. As the algorithm is basically checking if the number of weighted events
in each bin is the same for the simulated and recorded data. However, only the bins where the signal to background ratio
is high enough are considered.

Each bin thus contributes to the well-modelling score. If the bin is blinded, the contribution is zero. Otherwise, the
contribution for the bin is:

$$
    \text{Bin Contribution} = 1 - \frac{N_\text{Recorded} - N_\text{Simulation}}{\max(N_\text{Recorded}, N_\text{Simulation})}
$$

Where $N_\text{Recorded}$ and $N_\text{Simulation}$ are the number of weighted events in the bin for the recorded and
simulated data respectively.

The total well-modelling score is then the mean of the contributions for all the bins.



\subsection{Training times}

% Feynman diagram example
% \begin{equation}
%     \begin{tikzpicture}
%         \begin{feynhand}
%             \vertex[ringdot,label=above:{s}] (v) at (0,0) {};
%             \vertex[right=of v] (f1) ;
%             \vertex [above left=of v] (i1);
%             \vertex [below left=of v] (i2);
%             \propag [fermion, momentum'=\(p_1\)] (i1) to (v);
%             \propag [anti fermion,momentum'=\(p_2\)] (i2) to (v);
%             \propag [boson, momentum'=\(p_3\)] (v) to (f1);
%         \end{feynhand}
%     \end{tikzpicture}
% \end{equation}

% Bibliography
\printbibliography

\end{document}